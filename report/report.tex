% multiple1902 <multiple1902@gmail.com>

\documentclass[11pt]{report}
\usepackage{osexp}

\renewcommand{\CLASS}{92}
\renewcommand{\NUMBER}{09055029}
\renewcommand{\NAME}{Weisi Dai}

\begin{document}

\frontpage

\chapter{内核编译与系统调用}

    \section{实验目的}

        \begin{enumerate}
            \item 熟悉Linux源代码的目录结构, 掌握编译Linux内核的一般方法;
            \item 了解Linux中系统调用的实现, 学会编写简单的Linux系统调用.
        \end{enumerate}

    \section{实验内容}

        在 Oracle VM VirtualBox 虚拟机软件中创建一个虚拟机, 并安装 Linux 操作系统. 在虚拟机中编译内核, 并在内核中实现自定义系统调用.

    \section{实验环境}

        \subsection{宿主机}

            \begin{itemize}
                \item openSUSE Linux (Linux 3.1.10-1.9 --- 3.3.3-21 x86\_64)
                \item ArchLinux (Linux 3.2.7-1 --- 3.3.4-2 x86\_64)
                \item Oracle VM VirtualBox 4.1.8 OSE (build 75467) --- 4.1.14 (build 77440)
            \end{itemize}

        \subsection{虚拟机}

            \begin{itemize}
                \item Arch Linux (Linux 3.2 x86)
                \item GCC
                \item openssh
            \end{itemize}

    \section{实验步骤}

        \subsection{安装Linux操作系统}

            \subsubsection{创建虚拟机和安装操作系统}

                启动VirtualBox, 添加一台虚拟机, 名称为``OS\_Exp'', 操作系统设置为Arch Linux, 分配内存512MB, 创建新虚拟磁盘(8GB). 从光盘镜像 \verb|archlinux-2011.08.19-core-i686.iso| 启动.

                在引导菜单中, 选择``Boot Arch Linux'', 进入Arch Linux的终端. 按照提示, 执行 \verb|/arch/setup| 启动Arch Linux的安装程序. 选定安装源和时区后, 进入手工分区界面. 将8GB硬盘空间分为一个ext4分区, 并挂载到 \verb|/| .

                只选择安装 base 分组中的包, 开始安装系统. 软件包安装完成后, 进行安装后的设置. 修改 \verb|/etc/rc.conf| 的 62 行为 \verb|HOSTNAME="osexp"| , 并打开 eth0 介面的 DHCP 服务. 设置 root 帐号密码后创建 initcpio 文件, 安装 grub 引导器, 结束安装过程. 在终端中执行 \verb|reboot| 重启进入Arch Linux.

            \subsubsection{配置虚拟机}

            \begin{enumerate}
                \item 安装基本服务

                    修改 \verb|/etc/pacman.d/mirrorlist|, 启用在美国的一个源. 执行 \verb|pacman -Sy| 更新软件源缓存. 执行 \verb|pacman -S vim openssh| 安装 vim 编辑器和 openSSH 服务器, 期间 pacman 会要求先更新其自身.

                \item 配置网络

                    在虚拟机中执行 \verb|pacman -S net-tools| 安装 \verb|ifconfig| 命令. 在 VirtualBox 中给虚拟机的 NAT 增加一条从宿主机的 TCP 34567 端口到虚拟机的TCP 22的端口映射.

                \item 启动SSH服务器

                    修改 \verb|/etc/ssh/sshd_config| 文件, 允许 root 用户通过 ssh 登录. 执行 \verb|/etc/rc.d/sshd start| 启动虚拟机上的openSSH服务器. 此时可以从宿主机访问虚拟机上的SSH服务:

                    \begin{console}
[multiple1902 ~]$ ssh root@127.0.0.2 -p 34567
root@127.0.0.1's password:
Last login: Sat Feb 25 19:44:13 2012 from 10.0.2.2
[root@osexp ~]#
\end{console}
                \item 设置 SSHFS 文件系统

                    执行 \verb|pacman -S sshfs| 安装 FUSE 和 sshfs 程序. 写入启动 sshfs 的脚本:

                    \begin{console}
[root@osexp /]# cat > enable_sshfs
sshfs 10.0.2.2:/media/store/kernel/osexp /mnt
\end{console}
            \end{enumerate}

        \subsection{编译内核}

            \subsubsection{安装用于编译的包}

\begin{consolelong}
[root@osexp /]# pacman -S base-devel
:: There are 12 members in group base-devel:
:: Repository core
   1) autoconf  2) automake  3) binutils  4) bison  5) fakeroot  6) flex
   7) gcc  8) libtool  9) m4  10) make  11) patch  12) pkg-config

Enter a selection (default=all):
resolving dependencies...
looking for inter-conflicts...

Targets (21): cloog-0.17.0-1  gcc-libs-4.6.2-7  isl-0.09-1
              libltdl-2.4.2-3  libmpc-0.9-2  mpfr-3.1.0.p3-1  ppl-0.11.2-2
              rsync-3.0.9-2  abs-2.4.3-2  autoconf-2.68-2
              automake-1.11.3-1  binutils-2.22-4  bison-2.5-3
              fakeroot-1.18.2-1  flex-2.5.35-5  gcc-4.6.2-7
              libtool-2.4.2-3  m4-1.4.16-2  make-3.82-4  patch-2.6.1-3
              pkg-config-0.26-2

Total Download Size:    27.20 MiB
Total Installed Size:   115.49 MiB
Net Upgrade Size:       97.85 MiB
\end{consolelong}

            \subsubsection{编译内核}

            从\url{http://www.kernel.org}下载到Linux 3.2.12的源代码压缩包, 解压. 用Mercurial跟踪源代码修改.

\begin{console}
[multiple1902 linux-3.2.12]% hg init
[multiple1902 linux-3.2.12]% echo syntax:glob > .hgignore
[multiple1902 linux-3.2.12]% cat .gitignore >> .hgignore
[multiple1902 linux-3.2.12]% hg add . > /dev/null
[multiple1902 linux-3.2.12]% hg ci -m "vanilla linux 3.2.12"
\end{console}

            取消自动设置版本号后缀.

\begin{console}
diff -r 5da0810a54a4 scripts/setlocalversion
--- a/scripts/setlocalversion    Fri May 04 19:38:29 2012 +0800
+++ b/scripts/setlocalversion    Fri May 04 20:41:43 2012 +0800
@@ -9,6 +9,9 @@
 #
 #
 
+echo "-OSEXP"
+exit
+
 usage() {
     echo "Usage: $0 [--save-scmversion] [srctree]" >&2
     exit 1
\end{console}

            获取当前内核配置:

\begin{console}
[multiple1902 ~]$ ssh root@127.0.0.2 -p 34567
Last login: Sat Mar 10 21:07:04 2012 from 10.0.2.2
[root@osexp ~]# zcat /proc/config.gz > /mnt/linux-3.2.12/.config
\end{console}

            微调内核参数.

\begin{console}
[multiple1902 ~]$ make menuconfig
\end{console}

            启动交叉编译:

\begin{console}
[multiple1902 linux-3.2.12]% i386
[multiple1902@m-laptop linux-3.2.12]$ make -j5
\end{console}

            安装内核:
\begin{console}
[root@osexp linux-3.2.12]# make modules_install install
\end{console}

            生成对应的initramfs:

\begin{console}
[root@osexp mkinitcpio.d]# cat osexp.preset
# mkinitcpio preset file for the 'linux' package

ALL_config="/etc/mkinitcpio.conf"
ALL_kver="/boot/vmlinuz"

PRESETS=('default')

#default_config="/etc/mkinitcpio.conf"
default_image="/boot/initramfs-linux-osexp.img"
#default_options=""
[root@osexp mkinitcpio.d]# mkinitcpio -p osexp
==> Building image from preset: 'default'
  -> -k /boot/vmlinuz -c /etc/mkinitcpio.conf -g /boot/initramfs-linux-osexp.img
==> Starting build: 3.2.12-OSEXP
  -> Parsing hook: [base]
  -> Parsing hook: [udev]
  -> Parsing hook: [autodetect]
  -> Parsing hook: [pata]
  -> Parsing hook: [scsi]
  -> Parsing hook: [sata]
  -> Parsing hook: [filesystems]
  -> Parsing hook: [usbinput]
==> Generating module dependencies
==> Creating gzip initcpio image: /boot/initramfs-linux-osexp.img
11365 blocks
==> Image generation successful
[root@osexp mkinitcpio.d]#
\end{console}

            设置引导菜单:

\begin{console}
[root@osexp ~]# sed -n /OsExp/,+3p /boot/grub/menu.lst 
title  Arch Linux OsExp
root   (hd0,0)
kernel /boot/vmlinuz root=/dev/sda1 ro
initrd /boot/initramfs-linux-osexp.img
\end{console}

            检验内核:

\begin{console}
[multiple1902 report]$ ssh root@127.0.0.2 -p 34567
Last login: Sat Mar 10 22:14:47 2012 from 10.0.2.2
[root@osexp ~]# uname -r
3.2.12-OSEXP
\end{console}

        \subsection{添加系统调用}

            切换代码到syscall分支:

\begin{console}
[multiple1902 linux-3.2.12]% hg branch syscall
marked working directory as branch syscall
\end{console}

            添加相应代码:

\begin{consolelong}
# HG changeset patch
# User Weisi Dai <multiple1902@gmail.com>
# Date 1332565183 -28800
# Branch syscall
# Node ID 28d5a8f6e76795dc8e601d1ea7879877c41038c5
# Parent  2ecb6924a3608ed5ca33381731d30592b94ffeb8
probepid

diff -r 2ecb6924a360 -r 28d5a8f6e767 arch/x86/include/asm/unistd_32.h
--- a/arch/x86/include/asm/unistd_32.h    Fri Mar 23 22:41:24 2012 +0800
+++ b/arch/x86/include/asm/unistd_32.h    Sat Mar 24 12:59:43 2012 +0800
@@ -228,7 +228,7 @@
 #define __NR_madvise1        219    /* delete when C lib stub is removed */
 #define __NR_getdents64        220
 #define __NR_fcntl64        221
-/* 223 is unused */
+#define __NR_probepid       223
 #define __NR_gettid        224
 #define __NR_readahead        225
 #define __NR_setxattr        226
diff -r 2ecb6924a360 -r 28d5a8f6e767 arch/x86/kernel/syscall_table_32.S
--- a/arch/x86/kernel/syscall_table_32.S    Fri Mar 23 22:41:24 2012 +0800
+++ b/arch/x86/kernel/syscall_table_32.S    Sat Mar 24 12:59:43 2012 +0800
@@ -222,7 +222,7 @@
     .long sys_getdents64    /* 220 */
     .long sys_fcntl64
     .long sys_ni_syscall    /* reserved for TUX */
-    .long sys_ni_syscall
+    .long sys_probepid
     .long sys_gettid
     .long sys_readahead    /* 225 */
     .long sys_setxattr
diff -r 2ecb6924a360 -r 28d5a8f6e767 kernel/sys.c
--- a/kernel/sys.c    Fri Mar 23 22:41:24 2012 +0800
+++ b/kernel/sys.c    Sat Mar 24 12:59:43 2012 +0800
@@ -1917,3 +1917,16 @@
     return ret;
 }
 EXPORT_SYMBOL_GPL(orderly_poweroff);
+
+asmlinkage int sys_probepid(){
+    int i=0;
+    struct task_struct *p;
+    p=&init_task;
+    do{
+        printk("[probepid] PID = %d, Sys CPU Time = %d, User CPU Time = %d \n", 
+                p->pid, p->stime, p->utime);
+        i++;
+    } while ((p=next_task(p)) && (p!=&init_task));
+    return i;
+}
+
\end{consolelong}

        这个系统调用实现了在内核输出中打印所有进程的 PID , 系统 CPU 时间, 用户 CPU 时间等信息.

        \subsection{再次编译内核}

            启动交叉编译. 在这里, 笔者比较困, 决定趁编译时休息一会, 以下命令行在编译完成后调用笔者写的s2myself脚本向手机上发送一条短信, 报告编译完成:

\begin{console}
[multiple1902 linux-3.2.12]% i386
[multiple1902@m-laptop linux-3.2.12]$ time make bzImage modules -j5 ; \
    s2myself "Compilation Finished"
real    25m38.145s
user    68m48.341s
sys     8m9.888s
\end{console}

    \section{功能测试和运行结果}

            设计执行单元测试:

\begin{console}
[multiple1902 /t]% ssh root@127.0.0.2 -p 34567        
Last login: Sat Mar 24 10:01:10 2012 from 10.0.2.2
[root@osexp ~]# . enable_sshfs
multiple1902@10.0.2.2's password:
[root@osexp ~]# cd /mnt/test/
[root@osexp test]# cat syscall.c
#include "unistd.h"
#define __NR_probepid 223
long probepid(){
    return syscall(__NR_probepid);
}
int main(){
    probepid();
    return 0;
}
[root@osexp test]# ./a.out
[root@osexp test]# dmesg | tail
[   47.239686] [probepid] PID = 644, Sys CPU Time = 0, User CPU Time = 0
[   47.240357] [probepid] PID = 645, Sys CPU Time = 1, User CPU Time = 0
[   47.241016] [probepid] PID = 646, Sys CPU Time = 0, User CPU Time = 0
[   47.241541] [probepid] PID = 647, Sys CPU Time = 0, User CPU Time = 0
[   47.242032] [probepid] PID = 648, Sys CPU Time = 0, User CPU Time = 0
[   47.242683] [probepid] PID = 649, Sys CPU Time = 12, User CPU Time = 0
[   47.244048] [probepid] PID = 651, Sys CPU Time = 3, User CPU Time = 2
[   47.244646] [probepid] PID = 661, Sys CPU Time = 5, User CPU Time = 2
[   47.245299] [probepid] PID = 662, Sys CPU Time = 0, User CPU Time = 0
[   47.245750] [probepid] PID = 666, Sys CPU Time = 8, User CPU Time = 0
\end{console}

    \section{实验总结}

        在 Linux 3.2.12 中添加了系统调用, 并编写程序测试了它.

\chapter{动态模块与设备驱动}

    \section{实验目的}

        \begin{enumerate}
            \item 熟悉编写 Linux 中系统动态模块的一般原理;
            \item 掌握字符设备驱动的编写方法.
        \end{enumerate}

    \section{实验内容}

        编写一个字符设备驱动, 并基于它实现一个聊天程序.

    \section{实验思想}

        字符设备在文件系统中表现为一个块设备, 可以进行读写操作.

    \section{实验步骤}

        建立 module 代码分支:
        
\begin{console}
[multiple192 linux-3.2.12]% hg update
4 files updated, files merged, files removed, files unresolved
[multiple192 linux-3.2.12]% hg branch module
marked working directory as branch module
(branches are permanent and global, did you want a bookmark?)
\end{console}

            加入字符设备驱动代码. 
                驱动程序核心文件 \verb|chardev.c|:

\begin{consolelong}
#include <linux/kernel.h>
#include <linux/module.h>
#include <linux/fs.h>
#include <asm/uaccess.h>

int init_module(void);
void cleanup_module(void);
static int device_open(struct inode *, struct file *);
static int device_release(struct inode *, struct file *);
static ssize_t device_read(struct file *, char *, size_t, loff_t *);
static ssize_t device_write(struct file *, const char *, size_t, loff_t *);

#define SUCCESS 0
#define DEVICE_NAME "chardev"
#define BUF_LEN 80
static int Major;    
static int Device_Open = 0;
static char msg[BUF_LEN];
static char *msg_Ptr;

static struct file_operations fops = {
    .read = device_read,
    .write = device_write,
    .open = device_open,
    .release = device_release
};

int init_module(void)
{
    register_chrdev(222, DEVICE_NAME, &fops);
    return SUCCESS;
}

void cleanup_module(void)
{
    unregister_chrdev(222, DEVICE_NAME);
}

static int device_open(struct inode *inode, struct file *file)
{
    static int counter = 0;

    if (Device_Open)
        return -EBUSY;

    Device_Open++;
    msg_Ptr = msg;
    try_module_get(THIS_MODULE);

    return SUCCESS;
}

static int device_release(struct inode *inode, struct file *file)
{
    Device_Open--;        

    module_put(THIS_MODULE);

    return 0;
}

static ssize_t device_read(struct file *filp,  
               char *buffer,
               size_t length,    
               loff_t * offset)
{
    int bytes_read = 0;

    if (*msg_Ptr == 0)
        return 0;

    while (length && *msg_Ptr) {

        put_user(*(msg_Ptr++), buffer++);

        length--;
        bytes_read++;
    }

    return bytes_read;
}

static ssize_t
device_write(struct file *filp, const char *buff, size_t len, loff_t * off)
{
    memset(&msg[0],0,sizeof(msg));
    copy_from_user(msg, buff, len);
    printk("[chardev] [test] %s", msg);
    //return -EINVAL;
    return 0;
}
\end{consolelong}

        测试聊天程序 \verb|chat.c|:

\begin{consolelong}
#define MAXLEN 80
#define DEVPATH "/dev/chardev"

#include <stdio.h>
#include <string.h>
#include <unistd.h>
#include <sys/stat.h>
#include <sys/types.h>
#include <fcntl.h>
#include <sys/time.h>

pid_t fork(void);

char buff[MAXLEN]={0};

int update(){
    char buff_old[MAXLEN]={0};
    strcpy(buff_old, buff);

    int fd=0, ret=0;
    
    fd=open(DEVPATH, O_RDONLY);
    ret=read(fd,buff,MAXLEN);
    buff[ret]=0;
    close(fd);

    return strcmp(buff_old, buff);
}

void mysleep(){
    int millisecs, microsecs;
    struct timeval tv;

    millisecs=200;
    microsecs=millisecs*1000;

    tv.tv_sec=microsecs/1000000;
    tv.tv_usec=microsecs%1000000;

    select(0, NULL, NULL, NULL, &tv);
}

int main(){

    char myname[MAXLEN]={0};
    char inputbuff[MAXLEN]={0};

    printf("Please input your name: ");
    scanf("%s", myname);

    int pid;
    if((pid=fork())==0){   
        while(1){
            if(update()){
                printf("%s", buff);
            }
            mysleep();
        }
    }

    while(1){
        gets(inputbuff);
        if(strlen(inputbuff)){
            char cmd[MAXLEN+MAXLEN]={0};
            strcpy(cmd, "bash -c 'cat > ");
            strcat(cmd, DEVPATH);
            strcat(cmd, " <<EOF 2>/dev/null \n");
            strcat(cmd, myname);
            strcat(cmd, ": ");
            strcat(cmd, inputbuff);
            strcat(cmd, "\nEOF'");
            system(cmd);
        }
    }

    printf("%s", &buff);

    return 0;
}
\end{consolelong}

        这个设备占用了设备号 222, 支持写入和读出操作, 且最大长度为 80 字节.

        聊天程序利用循环结构从 \verb|/dev/chardev| 中拉取字符串, 若有更改则输出到屏幕上. 

        编写 Makefile:

        \begin{consolelong}
obj-m += chardev.o

all:
    rmmod chardev || true
    make -C ../ M=$(PWD) modules  && insmod ./chardev.ko

clean:
    make -C ../ M=$(PWD) clean

mknod:
    mknod /dev/chardev c 222 0

unittest:
    dmesg | tail
    sleep 1
    cat < dump > /dev/chardev || true
    sleep 1
    dmesg | tail
    sleep 1
    cat /dev/chardev

chat: chat.c
    gcc -o chat chat.c

trychat:
    ./chat || true
    cat /dev/chardev
\end{consolelong}

    \section{测试功能}
    
        创建设备结点, 并读出数据. ``The quick brown fox jumps over the lazy dog.'' 是 \verb|chardev| 模块内置的初始内容.

\begin{console}
[root@osexp chardev]# make mknod
[root@osexp chardev]# dd if=/dev/osexpdev bs=80 count=1
The quick brown god jumps over the lazy dog.
0+1 records in
0+1 records out
45 bytes (45 B) copied, 0.0010639 s, 42.3 kB/s
\end{console}

        在两个终端中测试执行聊天程序:

        \begin{console}
[root@osexp chardev]# ./chat
Please input your name: Alice
Heres to the crazy ones.
The misfits. The rebels. The troublemakers.
The round pegs in the square holes.
Bob: While some may see them as the crazy ones, we see genius.
Bob: Because the people who are crazy enough to think 
Bob:  they can change the world, are the ones who do. 
        \end{console}

        \begin{console}
[root@osexp chardev]# ./chat
Please input your name: Bob
Alice: Heres to the crazy ones.
Alice: The misfits. The rebels. The troublemakers.
Alice: The round pegs in the square holes.
While some may see them as the crazy ones, we see genius.
Because the people who are crazy enough to think 
 they can change the world, are the ones who do. 
        \end{console}

    \section{实验总结}

        完成了一个容量为 80 字节的字符设备的驱动, 并基于它编写了多人聊天程序.

\chapter{文件系统}

    \section{实验目的}

        \begin{itemize}
            \item 理解 Linux 内核模块的组织方式以及 ext4 文件系统的工作机理.
        \end{itemize}

    \section{实验内容}

        克隆 Linux 3.2.12 内核中的 ext4 文件系统.

    \section{实验步骤}

        为了创建 ext4 文件系统的代码拷贝, 进入 \verb|linux/fs| 目录, 将 \verb|ext4| 目录复制一份为 \verb|osexpext4|:

\begin{console}
[multiple1902 fs] hg cp ext4 osexpext4 
copying ext4/Kconfig to osexpext4/Kconfig
copying ext4/Makefile to osexpext4/Makefile
copying ext4/acl.c to osexpext4/acl.c
copying ext4/acl.h to osexpext4/acl.h
... ...
copying ext4/xattr_security.c to osexpext4/xattr_security.c
copying ext4/xattr_trusted.c to osexpext4/xattr_trusted.c
copying ext4/xattr_user.c to osexpext4/xattr_user.c
[multiple1902 fs] cd osexpext4
\end{console}

            将代码中的 \verb|ext4| 替换为 \verb|osexpext4|:

\begin{console}
[multiple1902 osexpext4] for i in `ls`; do
for> sed -i "s/ext4/osexpext4/g;s/EXT4/OSEXPEXT4/g" $i ;
for> done
[multiple1902 osexpext4]% for i in ext4*; do
for> hg mv $i $(echo $i | sed ``s/ext4/osexpext4/''); 
for> done
\end{console}

            设计 osexpext4 文件系统的 magic number, 写入 \verb|magic.h| 文件:

\begin{console}
diff -r be5b87a669d7 include/linux/magic.h
--- a/include/linux/magic.h     Thu Apr 26 21:38:46 2012 +0800
+++ b/include/linux/magic.h     Thu Apr 26 22:09:46 2012 +0800
@@ -22,6 +22,7 @@
 #define EXT3_SUPER_MAGIC       0xEF53
 #define XENFS_SUPER_MAGIC      0xabba1974
 #define EXT4_SUPER_MAGIC       0xEF53
+#define OSEXPEXT4_SUPER_MAGIC  0x8964
 #define BTRFS_SUPER_MAGIC      0x9123683E
 #define NILFS_SUPER_MAGIC      0x3434
 #define HPFS_SUPER_MAGIC       0xf995e849
\end{console}

            修改 \verb|fs/Makefile|, 加入 osexpext4 文件系统类型:

\begin{console}
diff -r be5b87a669d7 fs/Makefile
--- a/fs/Makefile    Thu Apr 26 21:38:46 2012 +0800
+++ b/fs/Makefile    Thu Apr 26 22:28:24 2012 +0800
@@ -68,6 +68,7 @@
 # We place ext4 after ext2 so plain ext2 root fs's are mounted using ext2
 # unless explicitly requested by rootfstype
 obj-$(CONFIG_EXT4_FS)        += ext4/
+obj-$(CONFIG_OSEXPEXT4_FS)        += osexpext4/
 obj-$(CONFIG_JBD)        += jbd/
 obj-$(CONFIG_JBD2)        += jbd2/
 obj-$(CONFIG_CRAMFS)        += cramfs/
\end{console}

            选择将 osexpext4 编译为模块:

\begin{console}
# HG changeset patch
# User Weisi Dai <multiple1902@gmail.com>
# Date 1335456748 -28800
# Branch fs
# Node ID ad690f2fde288055423d4004a77731799429848a
# Parent  a5737cd95b0dc6ba8aced00722615e9e5871e8e3
added osexpext4 to dotconfig

diff -r a5737cd95b0d -r ad690f2fde28 .config
--- a/.config	Fri Apr 27 00:11:30 2012 +0800
+++ b/.config	Fri Apr 27 00:12:28 2012 +0800
@@ -4811,6 +4811,10 @@
 CONFIG_EXT4_FS_XATTR=y
 CONFIG_EXT5_FS_POSIX_ACL=y
 CONFIG_EXT4_FS_SECURITY=y
+CONFIG_OSEXPEXT4_FS=m
+CONFIG_OSEXPEXT4_FS_XATTR=y
+CONFIG_OSEXPEXT4_FS_POSIX_ACL=y
+CONFIG_OSEXPEXT4_FS_SECURITY=y
 # CONFIG_EXT4_DEBUG is not set
 CONFIG_JBD=m
 # CONFIG_JBD_DEBUG is not set
\end{console}

        在 \verb|fs/Kconfig| 中引用 \verb|fs/osexpext4/Kconfig|:

        \begin{console}
# HG changeset patch
# User Weisi Dai <multiple1902@gmail.com>
# Date 1335459109 -28800
# Branch fs
# Node ID 87a9b1deaa201d83db1ebf1e6c801ea921bf8e01
# Parent  faf2193de3546156ec282a117e1a4a8c01edf1ea
sourced the osexpext4 Kconfig

diff -r faf2193de354 -r 87a9b1deaa20 fs/Kconfig
--- a/fs/Kconfig	Fri Apr 27 00:46:50 2012 +0800
+++ b/fs/Kconfig	Fri Apr 27 00:51:49 2012 +0800
@@ -9,6 +9,7 @@
 source "fs/ext2/Kconfig"
 source "fs/ext3/Kconfig"
 source "fs/ext4/Kconfig"
+source "fs/osexpext4/Kconfig"
 
 config FS_XIP
 # execute in place
        \end{console}

        受到包含文件的影响, 将 \verb|include/trace/events/ext4.h| 复制为 \\ \verb|include/trace/events/osexpext4.h|.

        此时可以编译内核了.

    \section{测试功能}

        在一个大小为 1MB 的文件中创建一个 osexpext4 文件系统, 查看它的十六进制结构, 并挂载这个文件系统, 向里面写入内容.

        创建文件:

\begin{console}
[root@osexp ~]# dd if=/dev/zero of=myfs bs=1M count=1
1+0 records in
1+0 records out
1048576 bytes (1.0 MB) copied, 0.00119143 s, 880 MB/s
\end{console}

        创建文件系统:

\begin{consolelong}
[root@osexp ~]# mkfs.ext4 myfs
mke2fs 1.41.14 (22-Dec-2010)
myfs is not a block special device.
Proceed anyway? (y,n) y
Filesystem label=
OS type: Linux
Block size=1024 (log=0)
Fragment size=1024 (log=0)
Stride=0 blocks, Stripe width=0 blocks
128 inodes, 1024 blocks
51 blocks (4.98%) reserved for the super user
First data block=1
Maximum filesystem blocks=1048576
1 block group
8192 blocks per group, 8192 fragments per group
128 inodes per group

Writing inode tables: done

Filesystem too small for a journal
Writing superblocks and filesystem accounting information: done

This filesystem will be automatically checked every 23 mounts or
180 days, whichever comes first.  Use tune2fs -c or -i to override.
\end{consolelong}

        查看超级块结构, 其中的 magic number 是 EF53H, 因此这是个 ext4 文件系统:

\begin{consolelong}
[root@osexp ~]# od -x myfs | head
0000000 0000 0000 0000 0000 0000 0000 0000 0000
*
0002000 0080 0000 0400 0000 0033 0000 03da 0000
0002020 0075 0000 0001 0000 0000 0000 0000 0000
0002040 2000 0000 2000 0000 0080 0000 0000 0000
0002060 86e3 4f99 0000 001b ef53 0001 0001 0000
0002100 86e3 4f99 4e00 00ed 0000 0000 0001 0000
0002120 0000 0000 000b 0000 0080 0000 0038 0000
0002140 0242 0000 0079 0000 fa85 b97d a6c0 3841
0002160 6297 38f8 a8ee 5e3e 0000 0000 0000 0000
\end{consolelong}

        将 EF53H 替换为 osexpext4 的 magic number:

\begin{console}
[root@osexp ~]# sed -i 's/'$(echo "\x53\xef")'/'$(echo "\x64\x89")'/' myfs
[root@osexp ~]# od -x myfs | head
0000000 0000 0000 0000 0000 0000 0000 0000 0000
*
0002000 0080 0000 0400 0000 0033 0000 03da 0000
0002020 0075 0000 0001 0000 0000 0000 0000 0000
0002040 2000 0000 2000 0000 0080 0000 0000 0000
0002060 7177 4f99 0000 0020 8964 0001 0001 0000
0002100 7177 4f99 4e00 00ed 0000 0000 0001 0000
0002120 0000 0000 000b 0000 0080 0000 0038 0000
0002140 0242 0000 0079 0000 7020 efe3 e232 ef4e
0002160 1886 8fd1 63eb 6914 0000 0000 0000 0000
\end{console}

        挂载 \verb|myfs| 文件并尝试写入内容:

\begin{console}
[root@osexp ~]# mount -t osexpext4 -o loop ./myfs /mnt
[root@osexp ~]# bash
[root@osexp ~]# cd /mnt
[root@osexp mnt]# ls
lost+found
[root@osexp mnt]# curl http://twitter.com/ > twitter_homepage 2>/dev/null
[root@osexp mnt]# head twitter_homepage  -n 5
<!DOCTYPE html>
<html data-nav-highlight-class-name="highlight-global-nav-home">
  <head>

    <title>Twitter</title>
[root@osexp mnt]# sha1sum twitter_homepage
1aac6156b7f90d6e5ade4b78a32925464e8ff395  twitter_homepage
\end{console}

    \section{实验总结}

        成功将 ext4 文件系统克隆为 osexpext4 文件系统, 并在过程中熟悉了内核的模块组织以及文件系统操作.

\end{document}
